\chapter*{ }
\section*{Streszczenie Pracy}


Celem niniejszej pracy dyplomowej było stworzenie systemu, który w połączeniu w symulatorem jazdy umożliwi testowanie algorytmów wizyjnych na zajęciach Systemy Wizyjne w Pojazdach Autonomicznych prowadzonych w ramach kierunku automatyka i robotyka na Akademii Górniczo-Hutniczej.

W części teoretycznej dokonano ewaluacji algorytmów wizyjnych zarówno korzystających z tradycyjnego przetwarzania obrazów cyfrowych, a także głębokouczonych sieci neuronowych. 

Po krótkiej analizie dostępnych symulatorów jazdy wybrano Euro Truck Simulator 2 czeskiego studia SCS Software. Jego przewaga nad innym programami dostępnymi na rynku polega na dużych możliwościach modyfikacji oraz interfejsu udostępnianego przed twórców.

Została stworzona aplikacja napisana z użyciem języka Python 3.7. Jej architektura jest wieloprocesowa, co oznacza, że każda z głównych funkcjonalności: przechwytywanie i analiza obrazu, generowanie sterowania i opcjonalne wyświetlanie wyników są realizowane równolegle, każda w osobnym procesie. Pozwoliło to na zwiększenie wydajności całego systemu, gdyż należy mieć na uwadze fakt, że oprócz działającej aplikacji na stacji roboczej jest uruchomiony symulator jazdy.
 
Działanie systemu zaprezentowano poprzez stworzenie 4 algorytmów wizyjnych.

\section*{Abstract of thesis}
